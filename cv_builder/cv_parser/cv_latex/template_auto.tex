\documentclass[10pt,a4paper]{altacv}
\geometry{left=1cm,right=9cm,marginparwidth=6.8cm,marginparsep=1.2cm,top=1cm,bottom=1cm}
\usepackage[utf8]{inputenc}
\usepackage[T1]{fontenc}
\usepackage[default]{lato}

\definecolor{VividPurple}{HTML}{2E64FE}
\definecolor{SlateGrey}{HTML}{2E2E2E}
\definecolor{LightGrey}{HTML}{666666}
\colorlet{heading}{VividPurple}
\colorlet{accent}{VividPurple}
\colorlet{emphasis}{SlateGrey}
\colorlet{body}{LightGrey}

\renewcommand{\itemmarker}{{\small\textbullet}}
\renewcommand{\ratingmarker}{\faCircle}

\DeclareUnicodeCharacter{0301}{}
\DeclareUnicodeCharacter{0302}{}
\begin{document}
\name{sami Fakhfakh}
\tagline{INGENIEUR EN INFORMATIQUE}
\photo{3cm}{image}
\personalinfo{
	\email{contact@sami-fakhfakh.com}
	\homepage{sami-fakhfakh.Com}
	\phone{+33 06 11 79 49 55}
	\location{2 Rue saint exupéry, Meudon la Forêt (92360) France}
	\dob{09/11/1995 A Lyon}
}
\begin{adjustwidth}{}{-8cm}
\makecvheader
\end{adjustwidth}
\marginpar{\vspace*{\dimexpr1pt-\baselineskip}\raggedright\cvsection{Compétences}
\cvskill{Java}{4}
\cvskill{Linux}{4}
\cvskill{Python}{3}
\cvskill{Ruby}{3}
\cvskill{C,C++}{3}
\cvskill{Php}{3}
\cvskill{Sql}{4}
\cvskill{Javascript}{3}
\cvskill{Docker}{2}
\cvskill{Aws}{2}
\cvsection{Certifications}
\begin{itemize}
\item Java 1z0-808 ( Oracle )
\item Linux Essentials 010-510 ( Linux Professional Institute )
\end{itemize}
\cvsection{LANGUES}
\cvskill{Français Bilingue DELF B2}{4}
\cvskill{Anglais TOEIC}{4}
\cvskill{Arabe Bilingue}{4}
}
\cvsection{EDUCATION}
\cvevent{}{Etudiant ingenieur informatique | Institut des sciences et techniques des yvelines - France }{2017 - 2020} {}
\begin{itemize}
	\item Ingénierie des Architectures Technologiques de l'Information et de la Communication. 
\end{itemize}
\cvevent{}{Licence fondamentale en informatique | Faculté des sciences de Sfax - Tunisie }{2014 - 2017} {}
\begin{itemize}
	\item Licence fondamentale en informatique 
\end{itemize}
\cvevent{}{BACCALAURÉAT | Lycée Habib Thameur Sfax - Tunisie }{2014} {}
\begin{itemize}
	\item baccalauréat sciences informatiques Mention Bien 
\end{itemize}
\cvsection{EXPERIENCE}
\cvevent{}{ALTERNANT | Contrat de professionalisation - BPCE SERVICES FINANCIERS - Paris }{SEPTEMBRE 2019 - SEPTEMBRE 2020} {}
\begin{itemize}
	\item Developpeur SUMMIT . 
\end{itemize}
\cvevent{}{STAGIAIRE | Stage - Crystalchain - Paris }{AVRIL 2019 - AOUT 2019} {}
\begin{itemize}
	\item Developpement d'un micro service de restitution de données pour une plateforme SaaS de traçabilité sur la blokchain \\ Automatisation de taches \\ Ruby on Rails \\ Python \\ Docker \\ Aws . 
\end{itemize}
\cvevent{}{Freelance | Freelance dans upwork }{MARS 2014 - NOV 2018} {}
\begin{itemize}
	\item Ayant du temps libre et assoiffé de savoir je me suis mis au freelance quand j'ai eu mon bac , cette experience m'as permise d'aquerir de nouvelles competances pratiques . 
\end{itemize}
\cvevent{}{STAGIAIRE | Stage de fin d’etudes - NextPlus - Sfax }{FEV 2017 - MAI 2017} {}
\begin{itemize}
	\item Durant ce stage j'ai pu creer une appplication en Java de détéction et de reconnaisance des plaques d'immatriculation des vehicules en temps réel.\\- Java\\- Python\\- OpenCv 
\end{itemize}
\cvsection{PROJETS}
\cvevent{}{Projet personnel Python | Script d'emailing en python d'envoie de mail en masse }{OCTOBTRE 2019} {}
\begin{itemize}
	\item Technologies: Python , Smtp 
\end{itemize}
\cvevent{}{Projet personnel Django | Application django de generation de cv dynamique }{SEPTEMBRE 2019} {}
\begin{itemize}
	\item Technologies: Python , Django , latex 
\end{itemize}
\cvevent{}{Projet universitaire NodeJs | Systéme de gestion des tâches et pense-bête  }{MAI 2019 - JUIN 2019} {}
\begin{itemize}
	\item Technologies: NodeJs , Html , css , javascript 
\end{itemize}
\cvevent{}{Projet freelance - upWork | Mise en place d’un script d’analyse technique et fondamentale ( Scrapping ) automatique , de décision et de simulation d’options binaires . }{JAN 2019} {}
\begin{itemize}
	\item Technologies: Python , Django , Selenium , plotly 
\end{itemize}
\cvevent{}{Projet universitaire Java | Developpement d’un jeu de dominos et triominos  }{JUIN 2018 - AOUT 2018} {}
\begin{itemize}
	\item Technologies : Java/Swing . 
\end{itemize}
\cvevent{}{Projet universitaire C | Developpement d’un mini systeme de gestion de fichier sous LINUX }{JUIN 2018 - AOUT 2018} {}
\begin{itemize}
	\item - Implementation d’un SGF sur un disque dur virtuel.\\- Developpement d’un ensemble de commandes operant sur un Shell virtuel permettant d’exploiter le SGF cree [C]. 
\end{itemize}
\cvevent{}{Projet universitaire C | Developpement d’un jeu de dominos et triominos. }{NOV 2017 - DEC 2017} {}
\begin{itemize}
	\item Technologies: C , sdl 
\end{itemize}
\cvevent{}{Projet universitaire C | Developpement d’un jeu de tarot . }{SEP 2017} {}
\begin{itemize}
	\item Technologies: C , sdl 
\end{itemize}
\cvevent{}{Projet PFE Java | Application desktop en Java pour la détection des véhicules avec la reconnaissance des plaques d’immatriculation. }{FEV 2017 - JUIN 2017} {}
\begin{itemize}
	\item Technologies: Java, Python, OpenCV, OCR, Machine Learning .  
\end{itemize}
\cvevent{}{Projet freelance - upWork | Applications (Web, Desktop et Mobile) pour la gestion des reservations et des offres de covoiturage et de colis-voiturage . }{JAN 2017 - FEV 2017} {}
\begin{itemize}
	\item Technologies: Java/Swing, J2EE EJB , Hibernate , Maven , JSP , REST API, MySQL, Git, Python, OCR 
\end{itemize}
\cvevent{}{Projet universitaire Java/SOAP | Application de gestion de comptes bancaires }{OCT 2015 - DEC 2015} {}
\begin{itemize}
	\item Exposition des webservices SOAP et leur consom- mation depuis une application Desktop développée en Java pour la gestion des comptes bancaires . 
\end{itemize}
\cvevent{}{Projet universitaire C | Jeu intellectuel ( Mastermind ) }{OCT 2014 - DEC 2014} {}
\begin{itemize}
	\item jeu Mastermind sur la console en C dans le cadre d’un projet academique. 
\end{itemize}
\end{document}
